
Die hier vorliegenden \LaTeX-Rezepte sind entstanden aus Fragen die im Verlaufe der Zeit an mich gerichtet wurden. Fehlen für Sie interessante Themen und/oder Problemstellungen, senden Sie mir bitte eine Mail\footnote{peter.kessler@id.ethz.ch} und ich werde diese möglichst bald in dieses Dokument aufnehmen.

Haben Sie selber schon Lösungen/Beispiele zu Themen und/oder Problemstellungen die hier noch nicht enthalten sind, so können Sie mir diese sehr gerne zustellen, damit ich diese (natürlich mit Quellenangabe) in dieses Dokument aufnehmen kann.

\LaTeX\ ist nie Liebe auf den ersten Blick. \LaTeX\ lernt man (mit zunehmender Erfahrung/Übung) lieben. Je mehr \LaTeX-Rezepte hier versammelt sind, desto leichter fällt es zukünftigen Studierenden sich in \LaTeX\ einzuarbeiten.

Die hier gezeigten \LaTeX-Rezepte beschränken sich weitestgehend auf die Standards von \LaTeX. Vieles kann den eigenen Vorlieben angepasst werden. Um nur einige wenige Punkte aufzuzählen:

\begin{itemize}
\setlength\itemsep{-1em}
\item \textbf{Kapitelnummern:} Im \LaTeX-Standard hat die letzte Zahl einer Kapitelnummer keinen Punk. Hätte man gern auf hinter der letzten Zahl einer Kapitelnummer einen Punkt, kann dies entsprechend angepasst werden.  
\item \textbf{Titelseite:} Im \LaTeX-Standard hat die Titelseite lediglich drei Einträge (Titel, Autor und Datum). Im Standard können auch mehr als ein Autor angegbeen werden. Es ist auch möglich eine komplett selber gestalltete Titelseite zu erstellen. 
\item \textbf{Andere Packages:} Der \LaTeX-Standard deckt schon sehr viel ab. Zu verschiedenen Themen (Tabellen, Grafiken, Verzeichnisse, etv.) gibt es weitere Packages, mit viel mehr Einstellmöglichkeiten als im \LaTeX-Standard.  
\item \textbf{Verzeichnisse:} Die hier erstellten Verzeichnisse entsprechen dem Default-\LaTeX-Standard. Man kann auch hier alles gemäss eigenen Wünschen anpassen. Sollen die Verzeichnisse anders formatiert werden, sollen sie andere Titel haben, oder sollen in den Verzeichnissen kürzere Bezeichnungen (z.B. Tab. anstelle von Tabelle, Abb. anstelle von Abbildung, ...) verwendet werden - alles lässt sich mit mehr oder weniger Aufwand anpassen.
\item \textbf{...}
\end{itemize}

Beispiele zu oben geführten Themen sind im ausführlichen \LaTeX-Tutorial auf meinem GitHub-Repo\footnote{\url{https://github.com/pkmlp}}  zu finden. 

Dieses \LaTeX-Dokument wurde erstellt mit TeXmaker und getestet auf MacOS, Linux und Windows 10.

Nicht vergessen: \LaTeX\ ist nie Liebe auf den ersten Blick. \LaTeX\ lernt man erst (mit zunehmender Erfahrung/Übung) lieben. Darum erstellen Sie am besten alle Ihre Dokumente mit \LaTeX. Starten Sie nicht erst mit Ihrer Bachelor-/Master-Arbeit mit \LaTeX.
