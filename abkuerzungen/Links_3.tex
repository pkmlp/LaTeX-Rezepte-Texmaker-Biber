
In einer wissenschaftlichen Arbeit dürfen oft wiederholte Begriffe durchaus abgekürzt werden. Die Verwendung von Abkürzungen sollte allerdings auf den Leserkreis abgestimmt sein. Neben den im Duden aufgeführten Abkürzungen dürfen auch Abkürzungen spezieller Fachtermini verwendet werden. Seien Sie sparsam mit Abkürzungen. Sie dürfen auf keinen Fall den Fluss Ihrer Arbeit stören.

Wird auf die hier vorgestellte Art mit Abkürzungen gearbeitet, so kümmert sich LaTeX um die korrekte Verwendung von Abkürzungen (einen Begriff bei erstmaliger Verwendung im Textabschnitt auszuschreiben und die Abkürzung in Klammer zu setzen, bei allen weiteren Malen nur noch die Abkürzung einzusetzen). Natürlich lässt LaTeX verschiedene Arten der Übersteuerung dieses Verhaltens zu.

Anders als Fachausdrücke, deren Bedeutung sich Fachfremden nicht auf den ersten Blick erschliesst, gehören allgemein übliche Ausdrücke wie 'z.B.' oder 'usw.' nicht in das Abkürzungsverzeichnis.
