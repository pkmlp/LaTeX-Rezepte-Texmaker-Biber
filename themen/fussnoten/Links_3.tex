
\textbf{Hinweis:} Da die Beispiele in diesem Dokument in Textboxes gesetzt sind, sind die Fussnoten hier mit Buchstaben anstelle von Zahlen (\LaTeX\ default). Die Art und Weise wie Fussnoten gesetzt werden, ist genau so wie auf der gegenüberliegenden Seite beschrieben. Sind die Fussnoten nicht in einer Textbox gesetzt, werden diese ganz normal im Dokument durchnummeriert.

\bigskip 
Zur Illustration hier das Gleiche auf einer 'normalen' Seite: 

Hier\footnote{Das ist die erste Fussnote} und hier\footnote{Das ist die zweite Fussnote} haben wir zwei unterschiedliche Fussnoten.

Soll von mehreren Stellen im Dokument auf eine Fussnote\footnote{Das ist eine Fussnote, auf die von mehreren Stellen verwiesen werden soll\label{ftn:multi}} verwiesen werden, z.~B. auch von hier\footref{ftn:multi} und auch noch von hier\footref{ftn:multi}, so muss die entsprechende Fussnote mit einem Label versehen werden und dann wird in den weiteren Malen auf dieses Label verwiesen.
