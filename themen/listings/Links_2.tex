
Der euklidische Algorithmus ist ein Algorithmus aus dem mathematischen Teilgebiet der Zahlentheorie. Mit ihm lässt sich der grösste gemeinsame Teiler zweier natürlicher Zahlen berechnen. Das Verfahren ist nach dem griechischen Mathematiker Euklid benannt, der es in seinem Werk „Die Elemente“ beschrieben hat.

\bigskip 

Der grösste gemeinsame Teiler zweier Zahlen kann auch aus ihren Primfaktorzerlegungen ermittelt werden. Ist aber von keiner der beiden Zahlen die Primfaktorzerlegung bekannt, so ist der euklidische Algorithmus das schnellste Verfahren zur Berechnung des grössten gemeinsamen Teilers: 

\bigskip 

In Python sieht der euklidische Algorithmus wie folgt aus:

\begin{quote}

\lstinputlisting[language=Python,
    basicstyle=\scriptsize,
    numbers=left,
    numberstyle=\tiny,
    stepnumber=2,
    numbersep=5pt,
    frame=single,
    framerule=0.1pt,
    showstringspaces=false,
    showspaces=false,
    showtabs=false,
    keywordstyle=\color{blue},
    commentstyle=\color{dkgreen},
    stringstyle=\color{mauve},
    backgroundcolor = \color{white}
]{themen/listings/listing_ggT.py}

\end{quote}
