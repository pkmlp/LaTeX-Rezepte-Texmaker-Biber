
In der Präambel:

\begin{verbatim}

% für mathematische Symbole und Formeln
\usepackage{amsmath, amssymb}

\end{verbatim}

\tcblower

Im Dokument: 

\begin{verbatim}

Da gibt es drei weitere Formeln, die jeder Obenstufenschüler kennt: 

\begin{align}
  (a+b)^{2}  &= a^{2} + 2ab + b^{2}   \label{eq:binome} \\
  (a-b)^{2}  &= a^{2} - 2ab + b^{2}   \nonumber         \\
  (a-b)(a+b) &= a^{2} - b^{2}         \nonumber
\end{align}

Die Binome \eqref{eq:binome}. Binome leiten sich von den Polynomen ab. 
Polynome sind mathematische Ausdrücke, deren Glieder durch Addition und 
Subtraktion verbunden sind. Diese Glieder können selber Produkte oder 
Ähnliches sein. Binome bezeichnen Polynome, die zwei Glieder besitzen. 
Entsprechend gibt es auch sogenannte Trinome, die drei Glieder besitzen 
und Monome, die nur aus einem Glied bestehen. 

\bigskip 

Der Binomische Lehrsatz liefert eine Darstellung für beliebig hohe Potenzen 
eines Binoms:

\begin{equation}
  (a+b)^n = \sum_{k=0}^n \binom{n}{k} a^{n-k} b^k
\end{equation}

\end{verbatim}
